	\chapter{概率、时间与黑洞的热}
\indent
   除了我已经讨论过的描述世界基本组成的主要理论以外,物理学仍然有一大堡垒与其他理论有所不同。一个简单的问题意想不到地产生了它:“什么是热?”

   截止到19世纪中叶时,物理学家们尝试着把热理解为一种叫做“热质”的流体;或者是两种流体:一种热的,一种冷的。这个想法后来被证明是错误的。詹姆斯·麦克斯韦与奥地利物理学家路德维希·玻尔兹曼最终理解了热的本质。他们的解释十分奇特出众而影响深远——并且带领我们进入了很大程度上尚未被探索的领域。

   他们的研究表明一个有热量的物体并不包含任何热质。它不过是由运动更剧烈的原子们所组成。原子与分子以及束缚在一起的小型原子团不停地运动着。它们飞驰着、振动着、碰撞着……冷空气是由运动更缓慢的原子——或者说,分子——所构成的。而热空气是由运动更激烈的分子构成的。这个理论简洁而美丽,但却不止于此。

   正像我们所熟知的那样,热量总是从热的物体传向冷的物体。比如一个放在一杯热茶中的冷茶匙会变热,而我们不在气温低时多穿一点就会很快丧失我们身体的热量并感到寒冷。那么为什么热量会从热的物体传向冷的物体而不是相反呢?

   这是一个关键的问题,因为这关系到时间的特性。在热量交换不发生或不明显的每个案例中,我们都能观察到系统的将来与过去十分相似。例如,热与太阳系中行星的运动几乎完全不相干,而事实上这样的运动可以反向的方式同等发生而不违反任何物理法则。然而,只要有热的存在,未来就不会与过去相同。比如说在没有摩擦的情况下,摆锤会不停地摆下去。如果我们将其拍摄下来并将录像倒放,我们将会发现这运动是完全可能的。但是如果存在摩擦,那么摆锤将会极轻微地传热给它的支撑物并失去能量、缓慢减速:摩擦生热。而且我们能立刻分辨出未来(沿摆锤减速的时间方向)与过去的不同。我们从未见过一个从静止状态开始通过从支撑物吸收热进而摆动的摆锤。未来与过去的差别只有存在热时才会出现。这种基本物理现象借助热量由热的物体导向冷的物体实现了未来与过去的分离。

   那么,再一次的,为什么随着时间流逝热从热的物体传向冷的物体而不是相反呢?

   玻尔兹曼发现了这个简单得令人惊奇的原因:纯粹的概率。

   玻尔兹曼的想法十分微妙,他引入了概率的理论。热并不是由于一条绝对的定律决定了它是由热的物体传向冷的物体的:热有如此的表现只是因为更大的可能性。这背后的原因是当一个处在较热物体中的快速移动的原子撞到一个较冷的原子时会更有可能传递给后者些许其自身的能量而不是相反。碰撞传递了能量但每次碰撞却并不同等地分配着能量。因此相互接触的物体的温度才会趋于一致。一个与较冷物体接触的较热物体是不可能通过它们间的联系而变得更热的:而这是极其不可能的。

   概率论向物理学核心的引入与在解释热动力学的基础方面的应用首当其冲地被认为是十分荒唐的。当时经常会看到没有人认真对待玻尔兹曼。1906.9.5,他在Trieste附近的Duino上吊自杀,永远也没有机会目睹他的理论收到普遍的承认。

   在第二课中我将量子力学如何预测每一小块物体的运动通过概率联系起来。这也将概率论放进了理论中央。但是玻尔兹曼考虑的关于热的本质的概率论与之有所不同。热力学中的概率在某种意义上与我们的无知紧密相连。

   也许我并不知晓有关概率论的相关内容,但我仍然可以确定一个对某事的或较大程度的可能性。例如,我不知道明天马赛港会下雨,也不会知道是晴天还是下雪,但是马赛港8月份下雪的概率很低。同样,对于大多数物体,我们有些许了解,但不完全知晓关于它们的状态的一切,我们只能根据概率进行预测。想象一个充满空气的气球。我可以测量它:测量它的形状、体积、压强、温度……但是气球内部的空气分子在其中快速移动,我不知道它们的确切位置。这阻止了我精确地预测气球的行为。例如,如果我解开这个结的封条,让它去它会缩小闹,冲和碰撞在一路,我将无法预测。因为我只知道它的形状、体积、压力和温度。气球的运动取决于内部分子的碰撞。然而,即使我不能准确地预测每一件事,我也能预测一件事或另一件事发生的可能性。例如,气球几乎不可能会飞出窗外绕着远处的灯塔旋转,然后降落到我的手上。有些时间更可能发生,而其他时间更不可能。

   同样,分子碰撞时热从较热的物体传递到较冷的物体的概率可以计算出比热向热物体传递的概率大得多。

   解释这些事实的科学分支被称为统计物理学。它的一个起自玻尔兹曼的成果解释了热和温度的概率性——那就是热力学。

   乍看之下,我们的无知昭示了物质的属性的想法似乎并不合理:热茶中的冷茶匙变热和气球被释放乱飞时。我们知道或不知道的事实与物理定律有什么关系?这个问题是合理的;而它的答案十分微妙。

   茶匙和气球的行为必须遵循完全独立于我们认知的物理定律。它们行为的可预测性或不可预知性与它们的精确状态无关;而与其相互作用的属性的有限集有关。这组属性取决于我们与茶匙或气球相互作用的具体方式。概率并不是指物质本身的进化。它涉及到与我们交互的那些特定数量的演变。再次地,我们用来组织世界的概念的深刻的本质关系显现了。

   冷茶匙在热茶中加热是因为茶和勺子相互作用。我们通过有限数量的变量,在无数的变量影响下,描述它们的行为。这些变量的值不足以准确预测未来的行为(比如气球的飞行),但足以预测勺子温度变化的最大可能性。

   我希望读者没有丧失对这些精妙的辨别的注重……

   现在,在二十世纪热力学(即关于热的科学)和统计力学(即不同运动的概率的科学)的课程已经扩展到了电磁和量子现象。然而,包含引力场的扩展被证明是有问题的。重力场在加热时的行为如何,仍是一个尚未解决的问题。

   我们知道在加热的电磁场中会发生什么:例如,在烤箱中有一种热电磁辐射可以烹饪馅饼,我们知道如何描述它。电磁波振动,随机共享的能量,我们将这团光子想象为在一个热气球中运动的分子。但是什么是热引力场呢?

   正如我们在第一节课中看到的,引力场即是空间本身。它影响时空。因此,当热量被扩散到重力场时,时间和空间本身必须振动……但我们仍然不知道如何描述这个井。我们没有方程来描述热时空的热振动。什么是振动时间?

   这些问题把我们带到时间问题的核心:时间的确切流向是什么?

   这个问题已经存在于古典物理学中,哲学家们在第十九、第二十世纪强调了这一点,但它在现代物理学中变得更为尖锐。物理学用公式来描述世界,它告诉事物如何随时间的变化而变化。但我们可以写公式,告诉我们如何在关系到他们的位置的不同,或是食物的味道关于黄油的变量函数。但时间似乎是流动的,而黄油或空间的位置并不流动。差别来自哪里?

   提出问题的另一种方法是问自己:什么是“现在”?我们说,只有现在的事物存在:过去不再存在,未来也不存在。但在物理学中,没有什么与“现在”的概念相对应。比较“现在”和“这里”。这里指定说话者的位置:两个不同的人在这里指向两个不同的地方。因此“这里”是一个词,它的意思是什么取决于它在哪里说。这种话语的术语是“索引”。“现在”也指说出这个词的瞬间,也归类为“索引”。但是没有人会梦想说“存在于这里”,而不存在于此的事物并不存在。那么为什么我们说“现在”存在,而其他事物不存在?“现在”是客观存在的东西吗?“流动”,使事物“一个接一个地存在”,还是只是主观的,像“这里”?

   这似乎是一个深奥的心理问题。但在现代物理学中已成为一个热点问题,因为狭义相对论表明,“现在”这个概念也是主观的。物理学家和哲学家得出的结论是,对于整个宇宙来说,一个普遍存在的概念是一个幻象,宇宙的“时间流”是一个不起作用的概括。当他的意大利朋友Michele Besso去世时,爱因斯坦写了一封感人的信给米歇尔的姐姐:“米歇尔已经一点点在我面前离开了这个陌生的世界。这意味着什么。像我们这样相信物理的人,知道过去、现在和未来之间的区别只不过是一种顽固的幻觉。

   幻想与否、是什么向我们解释了时间的流动与消逝呢?时间的流逝对我们所有人来说都是显而易见的:我们的思想和我们的语言都存在于时间之中;我们语言的结构需要时间——一件事物“是”或“曾是”或“将来是”。可以想象一个没有色彩的世界,没有物质,甚至没有空间,但很难想象一个世界没有时间。德国哲学家马丁·海德格尔强调了我们的“时间的栖居”。海德格尔对原始世界的描述是否有可能在世界的描述中消失?

   有些哲学家是海德格尔最忠实的追随者,他们得出结论:物理学不能描述现实的最基本的方面,并将其视为一种误导性的知识形式。但很多时候,在过去我们已经意识到,我们眼前的直觉并不精确的:我们将仍相信地球是平的且围绕太阳的。我们的直觉是在有限的经验基础上发展起来的。当我们向前看得更远一点时,我们发现世界并不像我们想象的那样:地球是圆的,开普敦的人脚高而头低。相信眼前的直觉,而不是理性、谨慎和聪明的集体检查,那就不是智慧:这是一个老人的假设,他拒绝相信他村外的伟大世界与他所知道的世界是不同的。

正如我们所看到的那样生动,我们对时间流逝的体验不需要反映现实的一个基本方面。但如果它不是根本性的,我们对时间流逝的生动体验,它来自哪里?

   我认为答案在于时间和热量之间的紧密联系。只有在有热的时候,过去和未来才会有区别。热与概率有关,而概率又与我们与世界其他地方的相互作用不符合现实的细节有关。时间的流动是从物理学中产生的,而不是在对事物的精确描述的背景下。它出现在统计和热力学的背景下。这可能是时间之谜的钥匙。现在不在客观上比“这里”的存在是客观存在的,但在微观世界的相互作用提示时间的现象出现在一个系统(例如,我们只有相互作用)通过无数的变量中。

   我们的记忆和意识是建立在这些统计上的现象。一个假设的超感官是不会有“流动”的时间:宇宙是一个单块的过去,现在和未来。但由于我们意识的局限性,我们只能觉察到世界的模糊景象,并活在时间之中。借用我的意大利编辑的话:“不明显比明显的要多得多。”从这有限的、模糊的焦点中,我们可以看到时间的流逝。懂了吗?不,它不是。还有很多东西需要理解。

时间处于引力、量子力学和热力学交叉点引起的问题的中心。我们仍然有一系列问题处在黑暗之中。如果我们可能已经开始了解量子引力,就会发现它结合了这三个谜题中的两个,但我们还没有一个理论能够把我们对世界的基本知识的三个部分结合起来。

   解决这个问题的一个小小的线索来自于物理学家Stephen Hawking完成的计算,这位物理学家虽然身体状况不好,但他却始终借助着机械的辅助勉力工作着。

   霍金利用量子力学成功地证明黑洞总是“热的”。它们像火炉一样散发热量。这是对“热空间”性质的第一个具体指示。从来没有人观察到这种热,因为它在目前观测到的黑洞中是微弱的,但霍金的计算是令人信服的,它以不同的方式重复,黑洞的热的现实被普遍接受。黑洞的热是物体上的量子效应,黑洞是自然界的引力。它是空间的单个量子,空间的基本粒子,振动分子,加热黑洞表面并产生黑洞热。这一现象涉及到量子力学、广义相对论和热科学的三个方面。黑洞的热就像物理学的Rosetta Stone,使用三种语言–量子、引力和热力学。它仍在等待我们来揭示时间的本质。

\noindent
