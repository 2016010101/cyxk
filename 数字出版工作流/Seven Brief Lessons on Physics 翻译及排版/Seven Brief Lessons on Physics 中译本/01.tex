	\chapter{最美丽的理论}
\indent

	阿尔伯特·爱因斯坦在年轻的时候有一年都在漫无目的地游手好闲。不幸的是,如果你不“浪费”时间的话,你不会获得任何东西——这一点是青年人的父母常常会遗忘的。他那时在帕维亚。他去那里与家人会合,这时的他不得不中断了德国的学业,无法在那里度过充实的高中生活。那时正是二十世纪初,意大利正在工业革命的初级阶段。他的父亲是一名工程师,正在潘丹(paduan)平原建立第一台电力工厂。阿尔伯特那时正在阅读康德的著作,有时还会旁听帕维亚大学的课程:这只是为了开心,而不需要在那里登记或者需要为考试着想。就是在这样的情况下,严肃的科学家才诞生。

   在登记入学苏黎世联邦理工学院后,他沉浸于物理学的世界中。几年后,1905年,他在当时最为著名的科学杂志,the Annalender Physik上发表了三篇文章。这三篇文章每一篇都是诺贝尔奖级别的文章。第一篇文章揭示了原子的存在性。第二篇文章为量子力学奠定了最初的基础,我将在下一章详细阐述。第三篇文章提出了他关于相对论最初的理论(今天称之为狭义相对论),这一理论揭示了对于不同的人来说,时间并不是以相同的速度流逝的:如果其中一人以很快的速度移动,那么两个孪生子会发现他们年龄不同。

   爱因斯坦一夜之间成为了一位著名的物理学家并得到了许多高校的聘书。但是还是有一些事情使他烦躁:尽管这一理论产生了轰动效应,但它并没有与我们所知的引力发生联系,即物理下落的原因。当他为他的理论写一篇总结时,他意识到这一点,并开始思考由现代物理学之父牛顿所提出的引力公式是否需要修改,以使其与新的相对性观念相吻合。他埋头于这一问题中,他花了十年时间去解决这一问题。这十年时间充满了枯燥的学习,尝试,错误,混乱,错误的文章,新奇的想法和被误解的观点。

   最终,在1915年的11月,他呈递了一份彻底解决这一问题的文章:一个关于引力的新理论,他称之为广义相对论——这是他的杰作,被伟大的俄罗斯物理学家列夫·朗道称为“最美丽的理论”。

   有很多的杰作使我们十分感动:莫扎特的安魂曲;荷马的奥德赛;西斯廷教堂;李尔王。去完全欣赏它们的优美之处往往需要很长的学徒期,但是这种奖赏是纯粹的美丽——但是又不止于此,这一理论使我们有了一种崭新的审视世界的视角。爱因斯坦的宝石,广义相对论,是描述自然界秩序的杰作。

   我还记得我开始理解广义相对论的兴奋与激动。那是一个夏天,我正在卡拉布里亚的Condofuri的海滩上,沐浴在希腊地区地中海的阳光下,还有一年就要大学毕业。一个人只有在不被上学分心的时候,才能最佳地学习。我那时正在一本被老鼠咬破边缘的书的帮助下学习,这是因为在晚上我用这本书来堵住一个位于Umbrian山丘上的破旧房子的老鼠洞,从前我常常在这间房子里来逃避一些无聊的大学课程。我总是会从书中抬起头看远方波光粼粼的大海:我好像真正看见了爱因斯坦所想象的空间和时间的弯曲。这一切好像是魔法:好像是一个朋友在低声告诉我一个非凡的秘密,突然间揭开了现实的面纱,使之显露出一个更为简单也更为深刻的秩序。自从我们发现地球是圆的并且像一个陀螺一样不断自转,我们就已经发现现实并不是看上去那样:每次我们发现现实的一个新方面的知识,我们就会感受到一次深刻的情感体验。又一层面纱被我们撕掉了。

   但是在历史上超越我们的认知的一个接着一个的发现中,爱因斯坦或许是无可匹敌的。这是为什么呢?

   首先,一旦你理解了相对论,你就会发现它那优美的简洁性。我会总结这一观点。

   牛顿尽最大努力去解释物体要掉落和行星要公转的原因。他设想了一个使得所有物体聚集在一起的力,称之为引力。这个力在距离很远但是中间没有其它物体的物体之间发生作用的途径还是未知的——而且牛顿很谨慎地给出了一个假设。他想象物体在空间中移动,空间本身像一个非常大的空容器,大到将整个宇宙都囊括进入,在这个容器中的所有物体都各行其是,直到有力使它们的轨迹发生弯曲。这个牛顿发明的包括整个世界的容器——空间是由什么组成的,他自己也说不清楚。但是爱因斯坦出生后几年,两位伟大的英国物理学家迈克尔·法拉第和詹姆斯·麦克斯韦向牛顿的经典力学体系中添加了关键性的一部分:那就是电磁场。这个场是一个真实的实体,充斥空间,传播电磁波,可以像湖面一样振动,借此传递电磁力。由于爱因斯坦年轻的时候就被驱动他父亲所建立的发电厂中转子运动的电磁场所吸引,他不久后就理解了引力,像电场力一样,一定是被一个场所传播的:一个类比于电磁场的引力场一定是存在的。他的目标是理解引力场的工作规律,以及用数学方程式描述它。

   在那时他想到了一个非凡的想法,一个完全是天才的想法:引力场并不是充斥于空间的,引力场就是空间本身。这就是广义相对论的雏形。牛顿的空间,就是物体移动所有经过的空间,以及引力场是完全相同的一个事物。

   这是一个富有启迪的时刻。一个对于世界极为重要的简化:空间不再是和物质无关的事物,它也是一种组成这个世界的物质。它是一个可以弯曲的事物。我们并不是被装在一个僵硬的看不见的结构中:我们处于一个非常大的弹性蜗牛壳中。太阳使得它周围的空间弯曲,而地球也不是因为一种神秘的力才绕着太阳公转,它是因为它向太阳笔直前进,像一块大理石在漏斗表面滚动一样。在漏斗中央并不存在一个维持这一切的神秘的作用力;是被弯曲的空间导致了大理石的滚动。行星的绕日公转,以及物体的下落,都是因为空间被弯曲了。

   我们怎样才能描述空间的曲率呢?十九世纪最为杰出的数学家,有“数学王子”之称的卡尔·弗里德里希·高斯,已经提出了描述二维曲面,例如山丘表面,的数学方程式。之后,他让一个聪明的学生去将这一理论推广至三维乃至更多维空间中。回答这一问题的学生,伯纳德·黎曼,提出了一种对这一问题影响深远的基本原则,尽管看上去它完全是无用的。黎曼论述的结论是弯曲空间是被特定数学实体承载,这一数学实体我们今天称之为黎曼曲面,通常用R来标记。爱因斯坦提出了一个方程式,在这一方程式中R等价于物体的能量。这就是说空间在有物质存在的地方会发生弯曲。结论就是这样。(The equation fitsinto half a line, and there is nothing more)如果说这一等式解决了一半的问题,那么这里就没有其它的问题了。一个空间弯曲的观测结果变成了一个等式。

   但是,这一个方程式描述了整个宇宙。这一理论的丰富性在这里打开一连串幻觉似的的预言,这些预言像极了疯子的的胡言乱语,但是这些在之后都被证明是真的。

   首先,这一方程式描述了星体周围的空间是怎样弯曲的。因为曲率的存在,不仅仅行星绕着恒星公转,而且光线也不再沿着直线传播而是发生一定的偏转。在1919年,这一偏转被测量了出来,结果和理论符合的很好。但不仅仅空间会弯曲,时间也会弯曲。爱因斯坦预言在地球表面,相较于低处而言,在高处时间会流逝地更快。这也被实验所证实了。如果一个住在海平面高度的人和他住在山上的孪生兄弟相遇,他会发现他兄弟会比他稍稍老一点。而这才是刚刚开始。

   当一个大的星体烧尽了它所有的可燃烧的物质(氢)时,它就会坍塌。当剩余的部分不再能承载燃烧的热时,它就会由于自身的重量坍塌,直到它使得空间坍塌到产生一个真实的洞。这就是著名的黑洞。当我在大学学习这些时,它们仅仅被认为是这一复杂理论的可能预言。今天,他们被数以百计地观测出来,并被天文学家们仔细地研究。

   但这依旧不是全部内容。这个空间都可以膨胀或者收缩。而且,爱因斯坦的方程式揭示了空间不可能是静态的,它一定是在膨胀。在1930年,宇宙的膨胀被真正地观测出来。同样的一个方程预言这一膨胀应该是被一个年轻的非常小但非常热的宇宙爆发所引起的:现在我们称之为“大爆炸理论”。又一次,最初没有人相信这一理论,但是相关的证据一直积累到宇宙背景辐射——由于最初爆炸所产生扩散的微波——被直接观测到。爱因斯坦方程式所产生的预言又一次被证明是正确的。但是进一步,这一理论断言空间像大海的表面一样运动。这些引力波的影响在宇宙中的双星系统中被观测出来,又一次与这一理论的预言相吻合,甚至吻合到令人惊讶的一千亿分之一的数量级上。其它情况也是这样。

   简而言之,这一理论描述了一个多姿多彩并且令人惊奇的世界。在这里,宇宙爆炸,空间坍塌成为一个无底洞,时间在行星附近变得缓慢,无边无际的星际空间像大海表面一样波动……这所有的一切,都渐渐地在我那本被老鼠咬过的书中出现,但它们并不是一个精神错乱的白痴所讲的故事,也不是一个被卡拉比亚地区地中海的烈日和令人眩晕的大海所引发的幻觉。这一切都是现实。

   或者更好地说,现实的一小部分,只是现实所蒙上诸多面纱中的一小部分。这个现实似乎是由和构成我们梦境相同的事物所构成的,但是比我们朦胧的梦境更为真实。

   所有的这一切都是一个基本的直觉的结果:空间和引力场是同样的一个东西。尽管你们几乎一定不能去理解这一简单的方程式,但我还是要在这里写下它。或许任何读到这的人都能去欣赏它那完美的简洁性:

                                            $R_{ab}-1/2*Rg_{ab}=T_{ab}$

   就是这样。

   你或许,当然,为了掌握阅读和使用这一方程式需要去学习和理解黎曼的数学工具。这会耗费一些时间和精力。但是相对于欣赏贝多芬后期弦乐曲秘密的美丽,这都是次要的。在这两种情形下,奖赏都是纯粹的美丽,以及看待世界的一种崭新视角。

\noindent
\iffalse
	\graybox{.8}{.65}{
		\bc\textcolor{gray}{\Large{\sffamily Notation used in this document:}}\ec

		\textbf{Abbreviations:}

		EM: electro-magnetic,\\
		UV: ultra-violet,\\
		$\gamma$: gamma-rays,\\
		X: X-rays,\\
		$e^-$: electron,\\
		$\pi$: pion,\\
		$\mu$: muon,\\
		$\nu$: neutrino, etc.\vspace{2ex}

		\textbf{Units:}\\
		International System:\\
		eV: electron volts,\\
		J: Joules,\\
		C: Celsius,\\
		M: mega,\\
		G: giga, etc.\vspace{2ex}

		\textbf{Chemical symbols:}\\
		The elements of the periodic table.\vspace{2ex}

		\textbf{References:}\\
		There are internal (marked in \textcolor{red}{red}) and external (marked in \textcolor{blue}{blue}) references.\vspace{2ex}
	}
\fi
